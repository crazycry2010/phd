\chapter{总结与展望}
\label{cha:conclusion}

\section{总结}

本文研究了视觉物体的识别问题。卷积神经网络在视觉物体识别问题中取得了重大的突破,本文从卷积层、池化层和模型压缩与加速三个方面对卷积神经网络进行了改进,并针对交通标志识别问题对这三个方法进行了验证。本文的主要研究内容按章节总结如下:

%第二章提出了一种卷积层的改进方法:自适应卷积模块。卷积层是卷积神经网络最基本的组成结构,具有特征提取的功能。对卷积层的改进主要体现在如何设计一种有效的卷积结构,来增强卷积层的特征提取能力。近年来,很多优秀的网络结构与模型被提出并应用于物体识别任务。
第二章提出了一种卷积层的改进方法:具有组合卷积结构的自适应卷积模块(SAM)。组合卷积结构结合了Inception、Maxout、ResNet和NIN四个优秀网络模型的特点与优势。SAM以Inception结构为基础计算框架,整体结构由四条特征提取分支与一个选择器构成。Inception计算框架可以有效地平衡四条分支的参数规模和计算量,在特征表达和计算量之前取得一个平衡。四条分支分别对应了不同的特征提取函数,其中包括两条卷积特征提取分支,分别具有不同的深度和感受野;一条Maxout分支可以增强组合卷积结构的拟合能力;一条ResNet分支可以加快网络的收敛速度。最后我们通过一个选择器来融合多个通道的特征,同时控制特征的维度。Maxout分支与ResNet分支共享了两条卷积分支的参数,进一步控制了SAM结构的参数规模与计算量。我们希望通过SAM结构的简单堆叠来简化复杂卷积神经网络的设计过程,四条特征提取分支加选择器的冗余结构可以使得SAM在网络的不同位置优化成不同的形式。在CIFAR-10、CIFAR-100、MNIST和SVHN四个数据集上,我们对SAM进行了测试,分别取得了5.76\%,28.56\%,0.31\%和1.98\%的测试错误率。实验结果表明,使用SAM来搭建卷积神经网络,可以有效地简化网络的设计,同时保持卓越的网络泛化能力。

第三章提出了一种池化层的改进方法:随机区域池化(SAP)。随机区域池化将数据增广的思想扩展到特征层,即特征增广。SAP通过随机仿射变换对池化区域进行重采样,再在变换后的区域上进行池化操作(均值池化或最大池化),因变换后的区域往往处于非整数边界,因此需要采用双线性差值对池化区域的像素进行估计。SAP在不改变特征空间样本分布的情况下,扩展特征空间,增加特征的多样性,从而提高网络的泛化能力,使网络对特征的扰动更加鲁棒。在CIFAR-10,CIFAR-100,MNIST和SVHN四个数据集上,我们对SAP进行了测试,分别取得了5.57\%,27.59\%,0.29\%和1.71\%的测试错误率。实验结果表明,SAP在没有明显增加网络参数与计算量的基础上,可以有效地提高网络的识别率与泛化能力。

第四章提出了一种网络模型加速的方法:主导卷积核分解(DK分解)和知识与回归(KP训练)。DK分解是受矩阵低秩分解启发而提出的一种卷积参数压缩方法。DK分解将传统的卷积操作分解为特征提取和特征组合两个过程,在特征提取的过程中,对于每个输入图像,我们仅仅选择具有主导作用的一个或多个卷积核对其进行特征提取,在特征组合的过程中,我们允许所有的特征同时参与特征的组合。DK分解可以将卷积层的参数和计算量压缩为大约原来的12\%,极大地提高网络的预测速度。为了尽量弥补网络因压缩所造成的精度问题,我们提出了KP训练方法。将原网络(教师网络)部分隐层特征的泛化能力(知识)迁移到压缩后的网络(学生网络),让学生网络尽可能地去学习教师网络的特征表达能力和泛化能力。实验结果表明,知识预回归可以加快学生网络的收敛速度,同时提高学生网络的泛化能力。

第五章针对一个实际问题交通标志识别问题,对前三章所提出的方法进行了实验验证与对比分析,更进一步地,我们将三个方法相结合,应用于解决交通标志识别问题。组合卷积结构和随机区域池化可以有效地与其他卷积网络结构配合使用,提高网络的特征提取与网络泛化能力。本章我们结合两种方法设计了一个具有特征增广的自适应网络模型FASNet,在德国交通标志识别数据集GTSRB上取得了99.66\%的识别率。针对FASNet网络参数和计算量较大的问题,我们采用主导卷积核分解对网络进行了压缩,得到压缩后的快速网络模型FastNet,并采用知识预回归的方法对FastNet进行训练,最终取得了99.27\%的识别率。相对于原始的FASNet网络结构,FastNet具有近10倍的参数压缩率和预测速度提升。


\section{展望}

本文的研究工作在视觉物体识别任务上取得了一些进展,但是在一些研究内容上仍有深入挖掘的空间。

首先,对于基于组合卷积结构的自适应卷积模块,目前仅仅在SAM模块中引入了四条特征提取分支。随着卷积神经网络的飞速发展,优秀的有参考价值的模型会越来越多,将更多的模型结构组合进SAM是一个长期积累的过程。将这些已经被验证过的优秀结构有效地组合与应用,进一步简化复杂卷积神经网络的设计过程,使卷积神经网络得到更加广泛的应用。此外SAM自身的结构相对较为复杂,导致对于SAM的理解和可视化工作变得非常困难,而网络的可视化工作是一个理解和优化网络结构的有效手段。

其次,对于特征增广,这是一个提高网络泛化能力的简单有效方法,目前我们仅仅在池化层验证了特征增广的效果。对于卷积神经网络来说,它的一个主要的特点就是分层的特征表达,将特征增广应用于更多的特征层是一个值得研究的课题。此外对于随机区域池化,我们仅仅研究了具有旋转、平移和缩放的特征增广方式,我们相信更加丰富有效的增广可以进一步增加网络的泛化能力。

最后,对于网络加速,这是卷积神经网络模型从研究走入实际应用的一个必要环节。在主导卷积核分解的方法中,我们仅仅对卷积层进行了参数的压缩,这是因为在我们的应用场景中卷积层占据了几乎全部的参数。而对于其他分类任务,比如百万级别、甚至上亿级别的人脸分类问题,全连接层也会引入大量的参数,如何有效地对全连接层进行DK分解,是一个有有待研究的课题。基于矩阵低秩分解的压缩方法毕竟还是精度有损的压缩,尽管知识预回归弥补了一部分精度损失,但是精度无损的压缩方法仍然是很具有吸引力的一个研究课题。


