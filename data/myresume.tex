\begin{resume}

  \resumeitem{个人简历}

  1988 年 8 月 3 日出生于 黑龙江 省  佳木斯 市。

  2007 年 9 月考入 哈尔滨工业大学 大学 计算机科学与技术 系 计算机科学与技术 专业,2011 年 7 月本科毕业并获得 工学学士 学士学位。

  2011 年 9 月进入 清华 大学 计算机 系攻读 博士 学位至今。

  \researchitem{发表的学术论文} % 发表的和录用的合在一起

  % 1. 已经刊载的学术论文(本人是第一作者,或者导师为第一作者本人是第二作者)
  \begin{publications}
    \item  Wang Z Y, Deng Z D, Wang S Y. Accelerating Convolutional Neural Networks with Dominant Convolutional Kernel and Knowledge Pre-regression. European Conference on Computer Vision (ECCV 2016). Springer, 2016: 533-548.
    \item Deng Z D, Wang Z Y, and Wang S Y. Stochastic Area Pooling for Generic Convolutional Neural Network. In Proc. 22nd Conference on Artificial Intelligence (ECAI 2016),  The Hague, Netherlands, 29 August - 02 September 2016, Frontiers in Artificial Intelligence and Applications, vol. 285, pp. 1760 – 1761, 2016. (EI收录,检索号:20170803367829)
     \item Wang Z Y, Deng Z D, Wang S Y. SAM: A Rethinking of Prominent Convolutional Neural Network Architectures for Visual Object Recognition. In Proc. International Joint Conference on Neural Networks (IJCNN 2016), Vancouver, Canada, 24 - 29 July 2016. (EI收录,检索号:	20165203194322)
     \item Wang Z Y, Deng Z D, Huang Z. Traffic Light Detection and Tracking Based on Euclidean Distance Transform and Local Contour Pattern. In Proc. of 2013 Chinese Intelligent Automation Conference. Springer Berlin Heidelberg, 2013: 623-631. (EI 收录,  检索号:20134216853040)
      \item Deng Z D, Wang Z Y, and Wang S Y. Generic Convolutional Neural Network with Random Pooling Area. 6th International Workshop on Combinations of Intelligent Methods and Applications (CIMA 2016). 2016.
  \end{publications}

  % 2. 尚未刊载,但已经接到正式录用函的学术论文(本人为第一作者,或者
  %    导师为第一作者本人是第二作者)。

  % 3. 其他学术论文。可列出除上述两种情况以外的其他学术论文,但必须是
  %    已经刊载或者收到正式录用函的论文。
%  \begin{publications}
%    \item Wu X M, Yang Y, Cai J, et al. Measurements of ferroelectric MEMS
%      microphones. Integrated Ferroelectrics, 2005, 69:417-429. (SCI 收录, 检索号
%      :896KM)
%    \item 贾泽, 杨轶, 陈兢, 等. 用于压电和电容微麦克风的体硅腐蚀相关研究. 压电与声
%      光, 2006, 28(1):117-119. (EI 收录, 检索号:06129773469)
%    \item 伍晓明, 杨轶, 张宁欣, 等. 基于MEMS技术的集成铁电硅微麦克风. 中国集成电路,
%      2003, 53:59-61.
%  \end{publications}

%  \researchitem{研究成果} % 有就写,没有就删除
%  \begin{achievements}
%    \item 任天令, 杨轶, 朱一平, 等. 硅基铁电微声学传感器畴极化区域控制和电极连接的
%      方法: 中国, CN1602118A. (中国专利公开号)
%    \item Ren T L, Yang Y, Zhu Y P, et al. Piezoelectric micro acoustic sensor
%      based on ferroelectric materials: USA, No.11/215, 102. (美国发明专利申请号)
%  \end{achievements}

\end{resume}
